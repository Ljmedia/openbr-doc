% A Readymade beamer presentation template
% Version 1.1
% Relase date: May 2, 2010
% Released at http://www.stattler.com
% by Rifat Jahan

\documentclass[12pt]{beamer}
%\usecolortheme[named=green]{structure}
\mode<presentation> {
\usetheme{Madrid} % My favorite!
%\usetheme{Boadilla} % Pretty neat, soft color.
%\usetheme{default}
%\usetheme{Warsaw}
%\usetheme{Bergen} % This template has nagivation on the left
%\usetheme{Frankfurt} % Similar to the default with an extra region at the top.
%\usecolortheme{seahorse} % Simple and clean template
%\usetheme{Darmstadt} % not so good
% Uncomment the following line if you want page numbers and using Warsaw theme
% \setbeamertemplate{footline}[page number]
%\setbeamercovered{transparent}
\setbeamercovered{invisible}
% To remove the navigation symbols from the bottom of slides%
\setbeamertemplate{navigation symbols}{} 
}

\usepackage{graphicx}
%\logo{\includegraphics[height=0.6cm]{../../openbr.png}}
\title[OpenBR]{OpenBR -- Open Source Biometric Recognition}
\author[J. Klontz \& B. Klare \& M. Burge]{Josh Klontz \& Brendan Klare \& Mark Burge}
\institute[]
{
www.openbiometrics.org \\
\medskip
{\emph{openbr-dev@googlegroups.com}}
}
\date{\today}

\begin{document}

\begin{frame}
\titlepage
\end{frame}

\section{Introduction}
\begin{frame}
\frametitle{Motivation}
\begin{block}
{Why Open Source?}
Brendan's passionate speech on the need for open source biometrics software!
\end{block}
\end{frame}

\begin{frame}
\frametitle{What's in it?}
\begin{itemize}
\item Off-the-shelf algorithms
  \begin{itemize}
  \item Face Recognition
  \item Gender Classification
  \item Age Estimation
  \item Document Classification
  \end{itemize}
\pause
\item Tools for algorithm evaluation
  \begin{itemize}
  \item Standardized set of file formats
  \item Automatic plot generation
  \item Command line interface
  \end{itemize}
\pause
\item Software framework for algorithm development
  \begin{itemize}
  \item Simple C++ plugin API for implementing new algorithms
  \item Built-in support for common algorithm use cases
  \item Automatic packaging and deployment
  \item Source code hosted on GitHub!
  \end{itemize}
\end{itemize}
\end{frame}

\begin{frame}
\frametitle{References}
\footnotesize{
\begin{thebibliography}{99}
 \bibitem[Label1, 2010]{key1} Author's name (1987)
 \newblock Title of the paper.
 \newblock \emph{Journal Name} 55(4), 765 -- 799.
\end{thebibliography}
}
\end{frame}

\begin{frame}
\centerline{The End}
\end{frame}


% End of slides
\end{document} 